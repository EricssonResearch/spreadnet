\documentclass{article}

% Language setting
% Replace `english' with e.g. `spanish' to change the document language
\usepackage[english]{babel}

% Set page size and margins
% Replace `letterpaper' with `a4paper' for UK/EU standard size
\usepackage[letterpaper,top=2cm,bottom=2cm,left=3cm,right=3cm,marginparwidth=1.75cm]{geometry}

% Useful packages
\usepackage{amsmath}
\usepackage{graphicx}
\usepackage[colorlinks=true, allcolors=blue]{hyperref}

\usepackage[
backend=biber,
style=alphabetic,
sorting=ynt
]{biblatex}

\addbibresource{sample.bib}
\begin{document}

\section{Introduction}

In computer science, the shortest-path problem is one of the extensive research area, specifically in graph theory. An ideal shortest path is one that meets the least length requirements between an origin and a destination. The topic has many varied applications, leading to a boom in research on shortest-path algorithms. These applications include network routing protocols, route planning, traffic control, path finding in social networks, computer games, transportation systems, and many more. \cite{DBLP:journals/corr/MadkourARRB17}

Many classical algorithms like Dijkstra, A*, and bellman Ford can solve shortest Path problems using their own strengths and weaknesses. 

In recent times Data mining and pattern recognition have been successfully facilitated by the rise and application of neural networks. Deep learning is capable of processing structured data, such as speech, images, and natural language, which has led to significant improvements in processing speech, images, and natural language. The problem jumps in when we do not have structured data like social networks, knowledge graphs, complex file systems, economic networks, chemical molecules, and worldwide webs. Typical Neural Networks(NN) like Convolutional Neural Networks (CNN), and Recurrent Neural Networks (RNN) are not capable of handling unstructured data input properly because these networks stack the feature of nodes by specific order. \cite{Mendoza_2019}

In this kind of unstructured problem Graph Neural Networks (GNN) can outperform to extract information from the unstructured data. 

A graph neural network is proposed to collect and summarize the information from the graph Structure. \cite{DBLP:journals/corr/abs-1812-04202}
Graph Neural Networks (GNNs)  are deep learning models for graph structured data, which achieve state-of-the-art results for many graph-related tasks.\cite{data7010010}
GNNs can be classified into different categories depending upon their model architecture. \cite{DBLP:journals/corr/abs-1901-00596}
Such as Recurrent GNNs[], convolutional GNNs \cite{DBLP:journals/corr/HamiltonYL17}
\cite{DBLP:journals/corr/KipfW16}
Graph Auto-Encoders \cite{https://doi.org/10.48550/arxiv.1611.07308}
and Spatial-Temporal GNNs. \cite{Yan_Xiong_Lin_2018}.


\printbibliography


\end{document}