%%%%%%%%%%%%%%%%%%%%%%%%%%%%%%%%%%%%%%%%%
% Uppsala University Assignment Title Page
% LaTeX Template
% Version 1.0 (27/12/12)
%
% This template has been downloaded from:
% http://www.LaTeXTemplates.com
%
% Original author:
% WikiBooks (http://en.wikibooks.org/wiki/LaTeX/Title_Creation)
% Modified by Elsa Slattegard to fit Uppsala university
% License:
% CC BY-NC-SA 3.0 (http://creativecommons.org/licenses/by-nc-sa/3.0/)

%\title{Title page with logo}
%----------------------------------------------------------------------------------------
%	PACKAGES AND OTHER DOCUMENT CONFIGURATIONS
%----------------------------------------------------------------------------------------

\documentclass[12pt]{article}
\usepackage[english]{babel}
\usepackage[utf8x]{inputenc}
\usepackage{amsmath}
\usepackage{graphicx}
\usepackage{float}
\usepackage[colorinlistoftodos]{todonotes}

\begin{document}

\begin{titlepage}

\newcommand{\HRule}{\rule{\linewidth}{0.5mm}} % Defines a new command for the horizontal lines, change thickness here

\center % Center everything on the page

%----------------------------------------------------------------------------------------
%	HEADING SECTIONS
%----------------------------------------------------------------------------------------

\textsc{\LARGE Uppsala University}\\[1.5cm] % Name of your university/college
\includegraphics[scale=.1]{Uppsala_University_seal_svg.png}\\[1cm] % Include a department/university logo - this will require the graphicx package
\textsc{\Large Course name}\\[0.5cm] % Major heading such as course name
\textsc{\large Course code}\\[0.5cm] % Minor heading such as course title

%----------------------------------------------------------------------------------------
%	TITLE SECTION
%----------------------------------------------------------------------------------------

\HRule \\[0.4cm]
{ \huge \bfseries Title}\\[0.4cm] % Title of your document
\HRule \\[1.5cm]

%----------------------------------------------------------------------------------------
%	AUTHOR SECTION
%----------------------------------------------------------------------------------------

\begin{minipage}{0.4\textwidth}
\begin{flushleft} \large
\emph{Author:}\\
First name \textsc{Last name}\\ % Your name
\end{flushleft}

\end{minipage}\\[2cm]

% If you don't want a supervisor, uncomment the two lines below and remove the section above
%\Large \emph{Author:}\\
%John \textsc{Smith}\\[3cm] % Your name

%----------------------------------------------------------------------------------------
%	DATE SECTION
%----------------------------------------------------------------------------------------

{\large \today}\\[2cm] % Date, change the \today to a set date if you want to be precise

\vfill % Fill the rest of the page with whitespace

\end{titlepage}
% \addbibresource{Bibliography/Sources.bib}        % Selects the Bibliography file.



\begin{document}
%%%%%%%%%%%%%%%%%%%%%%%%%%%%%%%%%%%%%%%%

\input{Text/0-frontmatter}
\input{Text/1-Introduction}
%\input{Text/0.1-Big_Figures}                    % Change this to another line to move big figures.
\input{Text/2-Background}
\input{Text/3-Experiment}
\input{Text/4-Related Work}
\input{Text/5-Conclusion}

%\clearpage                                     % Sometimes you want the rest on separate pages.
\input{Text/Acknowledgements}                   % Comment out to exclude the Acknowledgements section

%%%%%%%%%%%%%%%%%%%%%%%%%%%%%%%%%%%%%%%%


% Bibliography
%--------------------
\bibliography{reports/bibliography/sources}
\printbibliography

%\clearpage                                     % Sometimes it is useful to have appendix on separate page.
%\onecolumn                                     % If you want 1 column for appendix.
%\input{Text/Appendix}                           % Comment out to exclude appendix

\end{document}

\end{document}
